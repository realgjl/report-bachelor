\section{Hypothesis} %4.1
\label{section4.1}
\subsection{Hardware Hypothesis} %4.1.1
\label{subsection4.1.1}

In this Chapter, we use the algorithms described in the Chapter~\ref{Chapter3} to test the well-known Nordic system. One-line diagram of the Nordic test system is  shown in Figure~\ref{4_1_1_nordic}.\\

\begin{figure}[htbp]
\centering
\includegraphics[width = .891\textwidth]{figure/4_1_1_nordic.png}
\caption{One-line diagram of the test system}
\label{4_1_1_nordic}
\end{figure}

Firstly, we assume g2 as our monitor. The monitor sends its frequency, also the frequency in the system, to the communication layer. Then, we choose five thermal generators  (g6, g7, g14, g15, g16) in the CENTRAL area. There are three reasons I choose these five generators that produce the power to the system when disturbances occur in the system. First of all, these are thermal generators not the generators in the NORTH area which are equipped with hydraulic turbines. We only have strict mathematical proof of thermal generators. Another reason is they have nice nominal power as shown the Table~\ref{nominalPower} below. The second reason is the difference of them is up to 900 MW which is good for researching the function of generators comprehensively. The last reason is that I collaborated with a PhD student. We researched the \sys{Nordic Test System} with different algorithms. He used distributed method and I used centralised method. For convenience, we used the same generators.\\


\begin{table}[htbp]
\centering
\begin{tabular}{cc}
Generator & Nominal Power (MW) \\
g1        & 760.0              \\
g2        & 570.0              \\
g3        & 665.0              \\
g4        & 570.0              \\
g5        & 237.5              \\
g6        & 360.0              \\
g7        & 180.0              \\
g8        & 807.5              \\
g9        & 950.0              \\
g10       & 760.0              \\
g11       & 285.0              \\
g12       & 332.5              \\
g13       & 0.0                \\
g14       & 630.0              \\
g15       & 1080.0             \\
g16       & 630.0              \\
g17       & 540.0              \\
g18       & 1080.0             \\
g19       & 475.0              \\
g20       & 4275.0             

\end{tabular}
\caption{Generators Nominal Power in Nordic.}
\label{nominalPower}
\end{table}

After choosing the generators, we need to select a breaker. The breaker is a generator that can be disconnected to the system. Thus, a breaker is one of the most important impact factors to the grid system. However, at first, I was not sure how to choose a suitable breaker but then I tried to disconnect all of them one by one without Secondary Frequency Control. Results are as Figure~\ref{4_1_1_without1}. 

\begin{figure}[htbp]
\centering
\includegraphics[width = .891\textwidth]{figure/4_1_1_without1.jpeg}
\caption{Simulation results with different breakers, without SFC.}
\label{4_1_1_without1}
\end{figure}

As we can see from Figure~\ref{4_1_1_without1}, g8, g17, g18 and g20 are extremely unstable before 150 seconds and g13 have a zero frequency (because it has a zero nominal power, (Table~\ref{nominalPower})). Thus, we choose a breaker out of them to make sure we have a smooth testing environment. \\


\begin{figure}[htbp]
\centering
\includegraphics[width = .891\textwidth]{figure/4_1_1_without2.jpeg}
\caption{MATLAB figure: choose g9 as breaker.}
\label{4_1_1_without2}
\end{figure}

Finally, we chose g9 because it has a large steady-state error, as shown in Figure~\ref{4_1_1_without2}. Thus, the turbines will send more power to the system. It is easier to see the changes of kp and ki and to plot a 2D diagram of them.

\subsection{Software Hypothesis} %4.1.2

\begin{figure}[htbp]
\centering
\includegraphics[width = .891\textwidth]{figure/4_1_1_without3.jpeg}
\caption{MATLAB figure: choose start time for SFC.}
\label{4_1_1_without3}
\end{figure}
Next hypothesis is about start time and end time. In the definition of Primary Frequency Control (PFC), its power balance will be restored at a lower or higher frequency. Thus, from Figure~\ref{4_1_1_without3}, we can observe that, at the 150th seconds, the frequency is stable. Thus, I choose the 150th second at the beginning of Secondary Frequency Control. In the definition of Secondary Frequency Control, the response time will take up to 15 minutes, thus, I choose the 900th second as my temporary end time. \\

Since it is a low time delay testing, we can assume that delay is 0.01 seconds. The reason for not choosing an ideal situation (i.e. delay = 0 sec) is that, in reality, there be no such a zero-delay scenario.\\

Next step, we start tuning kp and ki. Following the idea in Section~\ref{section3.4}, firstly, we assumed a large value as a limit value of kp. Detailedly, we assumed the range of kp and ki are both from 0.1 to 300.1 and their steps are both 50.0.  \\

Then, we used the designed MATLAB program to check whether the simulation results are acceptable. Detailedly, we set overshoot smaller than 0.2\% because it is required that the frequency error should be in the range of ±0.1 Hz from Section~\ref{subsection4.1.1} in the Nordic official document. \\

We also need to make sure the signal is really settled from 0.9998 to 1.0002 before the end time (i.e. the 900th second), thus, it is necessary to extend the simulation time and give the program more data to check whether it's settled before 15 minutes. Thus, we set 1000 sec as our end time.\\

Finally, the program sketched plots of all the acceptable results and gave related information in the legend bar.\\

\begin{figure}[htbp]
\centering
\includegraphics[width = .819\textwidth]{figure/4_1_2_a.png}
\longcaption{Low time delay tuning 1}{\label{4_1_2_a} Low time delay tuning 1: Disconnect generator g9 when delay is 0.01 sec, kp is from 0.1 to 300.1 (step: 50), ki is from 0.1 to 300.1 (step: 50), start time is the 150th sec, end Time is the 1000th sec, and required settling time is the 900th sec.}
\end{figure}

However, as you can see from Figure~\ref{4_1_2_a}, only two simulations are acceptable. The maximum value of kp is 100.1 and the maximum value of ki is 0.1.  \\

According to the regulations from the last chapter, we need to use bisection method to increase the number of effective amplifier factor (i.e. kp and ki).\\

For instance, the new range of kp is between 0.1 and 150.1 and the new range of ki is between 0.1 and 50.1. To make more accurate results, we need to decrease the step at the same time. We set step be 10.0. \\ 

Finally, we have another plot as shown in Figure~\ref{4_1_2_b}. \\

\begin{figure}[htbp]
\centering
\includegraphics[width = .819\textwidth]{figure/4_1_2_b.png}
\longcaption{Low time delay tuning 2}{\label{4_1_2_b} Low time delay tuning 2: Disconnect generator g9 when delay is 0.01 sec, kp is from 0.1 to 150.1 (step: 10), ki is from 0.1 to 50.1 (step: 10), start time is the 150th sec, end Time is the 1000th sec, and  required settling time is the 900th sec.}
\end{figure}


Apparently, we have more acceptable results than the last simulation because we shrink the step of kp. We find that the simulations are unacceptable if kp equals to 140.1 or 150.1. Thus, we can finally fix the range of kp between 0.1 and 140.1 and keep the step of kp to 10. We can set ki in a range of 0.1 and 10.1 with the reason above. However, it is possible that the maximum value of ki is still 0.1 if we set the range of ki from 0.1 and 10.1. If so, then the simulations are meaningless. \\

Thus, we need to apply bisection method on the range of ki and find the meaningful maximum ki. Detailedly, we range kp from 0.1 to 140.1 and set ki equals to 5.1. The purpose of doing this is to find whether there are acceptable results if ki is in the middle of 0.1 and 10.1.\\

The filtered results are as Figure~\ref{4_1_2_c}. \\ 

\begin{figure}[htbp]
\centering
\includegraphics[width = .819\textwidth]{figure/4_1_2_c.png}
\longcaption{Low time delay tuning 3}{\label{4_1_2_c} Low time delay tuning 3: Disconnect generator g9 when delay is 0.01 sec, kp is from 0.1 to 140.1 (step: 10); ki is 5.1, start time is the 150th sec, end Time is the 1000th sec, and required settling time is the 900th sec.}
\end{figure}

The result shows there are acceptable simulations if ki equals to 5.1. \\

Thus, we can set the range of kp between 0.1 to 140.1 (step: 10.0), the range of ki between 0.1 and 10.1 (step: 1.0). The controller will start from 150 seconds and will end at 1000 seconds. We hope that the signal can be settled before 900 seconds.